\section{Constraint Generation Scheme}
\label{sec:cg}
Owing to the large number of constraints in the stronger models, LP($\mathcal{X}_3$) and LP($\mathcal{F}_3$) are impractical for computing lower bounds, even in fairly small instances.
The main idea of how to tackle larger instances and thereby make these models more useful in practice is to solve a relaxation of the model where some of the constraints are omitted.
It is assumed that the omitted constraints are often satisfied in solution of the relaxed problem, without being explicitly included in the model.
Relaxed constraints that are violated in the obtained solutions can be dynamically added to the model and the whole process is repeated, until some termination criteria are fulfilled.
This approach is known as a \emph{constraint generation scheme}.

\subsection{Implementation}%

According to experimental results presented in the next section, there is a trade-off between runtime and strength of the studied models.
LP relaxations of models $\mathcal{F}$ are stronger but slower than their counterparts $\mathcal{X}$.
The disproportion in runtime is most considerable in $\mathcal{F}_3$ and $\mathcal{X}_3$ whose strength is achieved because of the valid inequality (\ref{con:vi:sumFImpSumY}). 
%Experimental evaluation from the next section reveals that the valid inequality (\ref{con:vi:sumFImpSumY}) makes $\mathcal{X}_3$ very strong, in fact its LP relaxation often gives an integral solution.
%To the contrary, the other valid inequalities (\ref{con:vi:f2dest}) and (\ref{con:vi:xImpSumF}) rarely improve the LP bounds, and moreover increase the runtime.
For these practical reasons, we consider the model $\mathcal{X}_3$ for constraint generation.
Also note that LP($\mathcal{X}_1$) has the shortest runtime but not much shorter than LP($\mathcal{X}_2$).
The idea is therefore to solve LP($\mathcal{X}_2$) and generate only those remaining constraints that are violated.

%For each $(s,t)\in S$, constraints (\ref{con:mf:flowNormal})-(\ref{con:mf:fcap})form a classical maximum flow formulation which can be solved very quickly.
Consider $X$ fixed.
Whether there exists an $f\in\left[0,1\right]^{A\times S}$ satisfying (\ref{con:mf:flowNormal})-(\ref{con:mf:fcap}) can be checked by solving a maximum flow problem for each $(s,t)\in S$.
Observe that according to (\ref{con:mf:fcap}), $X^s$ plays the role of a capacity vector in the network $G_{X^s}=(V,E,s,t,X^s)$ in which we want to find the maximum flow, and thereby verify whether the flow conservation constraints are already satisfied even though they are not explicitly included in the model.
The formulation can also be extended by (\ref{con:vi:sumFImpSumY}).
Unfortunately, including (\ref{con:mf:fsym}) is slightly problematic here, because it contains variables of a different commodity, namely $(t,s)$.
It is not possible to simply extend the model by these constraints, because a potential violation of flow conservation for the commodity $(t,s)$ would not be discovered, as there are no corresponding capacity constraints.
Nevertheless, this can be easily resolved by combining the two commodities in one maximum flow formulation.
As $s$ and $t$ are now constants, let $f_{ij}=f^{st}_{ij}$.

This extended maximum flow formulation denoted as 2MF is constructed as follows:
\newline
\newline  
%%%%%%%%%%%%%%%%%%%%%%%%%%%%%%%%%%%%%%%
%                                     %
%          Model 2MF                  %
%        (13a) - (13f)                %
%                                     %
%%%%%%%%%%%%%%%%%%%%%%%%%%%%%%%%%%%%%%%
\begin{subequations}\label{eq:mf}
\begin{align}
\label{objective:maxf} &\max\sum\limits_{i \in V_s} f_{si} \\ 
\text{s.t.}  \notag   \\
\label{con:maxf:flowNormal}  \sum\limits_{\substack{ j\in V_i}}\left(f_{ij}-f_{ji}\right) & = 0 & i\in V\setminus\{s,t\},\\
\label{con:maxf:fcapst}   f_{ij} &\leq \min\{X_{ij}^s, X_{ji}^t\}     &  (i,j)\in A,  \\ 	 
\label{con:maxf:sumFImpSumYst} \sum\limits_{k\in V_i:p_{ik}\geq p_{ij}}f_{ik} & \leq \sum\limits_{k\in V_i:p_{ik}\geq p_{ij}}  \pi^{s}_{ik} & i,j\in V,\\
\label{con:maxf:sumFImpSumYts} \sum\limits_{k\in V_i:p_{ik}\geq p_{ij}}f_{ki} & \leq \sum\limits_{k\in V_i:p_{ik}\geq p_{ij}}  \pi^{t}_{ik} &  i,j\in V,\\
\label{con:maxf:fdim}f&\in\left[0,1\right]^{A}. 
\end{align}~
\end{subequations}  
  
This model comprises of constraints for $\{s,t\}\in\check{S}$ that are in $\mathcal{X}_3$ but not in $\mathcal{X}_2$.
Note that the right-hand sides in (\ref{con:maxf:fcapst})-(\ref{con:maxf:sumFImpSumYts}) are not variables.
These values are input data obtained by solving LP($\mathcal{X}_2$).
The problem modelled by this formulation can be understood as a maximum 2-commodity network flow, where the source of the first commodity is the target of the second commodity and vice versa.
Furthermore, the arcs have different capacities for the two commodities, even though it is required that the amount of one commodity passing through an arc $(i,j)$ equals the amount of the other commodity passing through $(j,i)$.
Due to the flow symmetry (\ref{con:mf:fsym}), we can always substitute $f^{ts}_{ji}$ with $f^{st}_{ij}$, and it is done so in (\ref{con:maxf:fcapst}) and (\ref{con:maxf:sumFImpSumYts}).
 
Based on these observations we build the constraint generation procedure.
First, we solve LP($\mathcal{X}_2$) and obtain vectors $x$ and $y$.
For each $(s,t)\in \check{S}$ we then check whether the relaxed constraints (\ref{con:mf:flowNormal})-(\ref{con:mf:fsym}) together with (\ref{con:vi:sumFImpSumY}) can already be satisfied by solving 2MF.
In case the objective value is 1, the relaxed constraints are satisfied.
Otherwise, we record the pair $(s,t)$, and after processing all pairs, relaxed constraints for a selection of the recorded pairs are added to the original model.
This augmented model is then solved again yielding new input values for the next iteration of the verification by 2MF.
The whole process is repeated until there are no violated flow constraints for any $(s,t)$ pair.
Algorithm \ref{alg:cg} describes the process more formally.

%%%%%%%%%%%%%%%%%%%%%%%%%%%%%%%%%%%%%%%
%                                     %
%         Algorithm 1                 %
%                                     %
%%%%%%%%%%%%%%%%%%%%%%%%%%%%%%%%%%%%%%%
\begin{algorithm}
\KwData{SMT instance}
\KwResult{Solution to LP($\mathcal{X}_3$)}
$Q'\leftarrow\emptyset$\tcp*{$\{s,t\}$ pairs of which constraints are to be added to $\mathcal{X}_2$}
\Do {$Q'\neq\emptyset$}{
	$Q\leftarrow\emptyset$\tcp*{Violated $\{s,t\}$ pairs} 
	\tcc{Solve the instance using $\mathcal{X}_2$ extended by constraints from 2MF associated with $\{s,t\}$ pairs in $Q'$}
	$(x,y)\leftarrow\text{solve($\mathcal{X}_2$,$Q'$.)}$\;
	\For{$\{s,t\}\in \check{S}$} {
		\tcc{Find the objective value of the extended maximum flow problem instance for given commodity $(s,t)$ using the capacities obtained earlier.}
		$v^*\leftarrow \text{solve(2MF,$s$,$t$,$x$,$y$)}$\;
		\If{$v^* < 1$} {
			$ Q\leftarrow Q\cup \{\{s,t\}\}$\;
		}
	}
	Select some subset $Q'\subseteq Q$ of the violated $\{s,t\}$ pairs according to a given strategy where $Q'\neq\emptyset$ if $Q\neq\emptyset$.
}
 ~\newline
 \caption{Constraint generation}
\label{alg:cg}
\end{algorithm}

\subsection{Strategies for constraint generation}

There are various strategies for selection of the violated flow constraints to be added to the original model.
Here are described those that we consider in the experimental part of this work.

\subsubsection{Add all}

One of the most basic strategies adds constraints of all violated $\{s,t\}$ pairs.
This approach turns out to be less practical, because after the first iteration, most of the $\{s,t\}$ pairs violate the 2MF constraints, and we end up adding too many of them which usually leads to a very long runtime of the next iteration.

\subsubsection{Add the first $k$}

Another simple strategy adds the first $k$ found $\{s,t\}$ pairs violating the flow constraints.
Even though it is not necessary to solve 2MF for all possible pairs of destinations, the runtime does not decrease noticeably.
Like in the previous case, this strategy does not exploit the structure of the graph induced by the violated $\{s,t\}$ pairs in any way.

\subsubsection{Add the best $k$}

This strategy is based on the assumption that addition of constraints for $\{s,t\}$ pairs for which the maximum flow value is lower, and are thus "more violated", is more likely to cause satisfaction of a larger number of constraints associated with other pairs of destinations that are violated but not yet included in $Q'$.

\subsubsection{Add maximum matching}

Another assumption that can be made is that addition of constraints for a given $\{s,t\}$ is likely to cause satisfaction of constraints associated with an adjacent pair $\{t,u\}$.
Thus, we are looking for a (maximum) matching $M$ in the weighted graph $(D,\check{S},w)$, where the weight function $w:\check{S}\to\mathbb{R}$ can be defined in several ways.
Probably the most intuitive is to use $v^*$ from Alg. \ref{alg:cg}.
Another suggestion is to define $w(\{s,t\})=p_{st}$, which is based on the assumption that a path between two distant nodes is long and addition of associated constraints may affect more edges.

%
%subsubsection{Add maximum spanning tree}
%
%\subsubsection{Heuristics}
%
